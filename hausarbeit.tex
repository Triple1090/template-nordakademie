\documentclass[11pt,a4paper,toc=bibliography,toc=listof,titlepage=firstiscover]{scrreprt}
\usepackage[utf8]{inputenc}
\usepackage{blindtext}
\usepackage{graphicx}
\usepackage[T1]{fontenc}
\usepackage{setspace}
\usepackage{pdfsync}
\usepackage{blindtext}
\usepackage[ngerman]{babel}
\usepackage{booktabs}
\usepackage{xcolor}
\usepackage{multirow}
%Schriftarten hier definieren
\usepackage{lmodern}
%\usepackage{tgheros}
%\renewcommand*\familydefault{\sfdefault}
%\usepackage{tgpagella}
\usepackage[babel,german=guillemets]{csquotes}
\usepackage[style=ieee-alphabetic]{biblatex}
\usepackage{hyperref}
\usepackage{csquotes}
\addbibresource{literatur.bib}
\hypersetup{colorlinks=true, linkcolor=black, urlcolor=black, citecolor=black}
%Abstand der Einträge im Literaturverzeichnis als em
\setlength\bibitemsep{0.75em}
%Kein Einzug bei einem neuen Absatz
\setlength{\parindent}{0mm}
%Inhaltsverzeichnis ohne eigene Seitenzahl
\AtBeginDocument{\addtocontents{toc}{\protect\thispagestyle{empty}}} 
\usepackage[left=3cm,right=2.5cm,top=2.5cm,bottom=2cm]{geometry}

%Daten des Deckblatts
  \titlehead
  {
    {
      \begin{center}
        \vspace{2.5cm}
        \includegraphics[scale=0.3]{resources/nak-logo.png}
      \end{center}
    }
  	%M.Sc. Angewandte Informatik
  }
  \subject{Hausarbeit}
  \title{Fallbeispiel Versicherung}
    \subtitle{Vorlesung Data Analytics}
    \author{Anna Hoege \and Tjark Radewaldt}
  \date{\small{Hamburg \& Hannover, den \today}}
  \publishers{Betreut durch Prof. Dr. Michael Schulz}
\begin{document}
%Titelseite erzeugen
\maketitle
%Der Nachfolgende Text wird 1,5-zeilig gesetzt
\onehalfspacing
%Römische Ziffern für Abbildungsverzeichnis
\pagenumbering{Roman}
\setcounter{page}{0}
%Inhaltsverzeichnis erzeugen
\tableofcontents
\clearpage
%Abbildungs und Tabellenverzeichnis
\listoftables
\listoffigures
%ToDo --> Abkürzungsverzeichnis
\clearpage
\pagenumbering{arabic}
\setcounter{page}{1}
%Datei aus dem Ordner /chapter/ einbinden. So kann man mehrere Textdateien zu einem großen Dokument zusammenfügen.
\setcounter{chapter}{0}
\include{chapter/10_kapitel-1.tex}
%Literaturverzeichnis einfügen
\renewcommand*{\UrlFont}{\rmfamily}
\printbibliography
\include{chapter/99_eidesstattliche-erkläerung.tex}
\end{document}
